\chapter{接口设计}
\section{外部接口}
在本项目中管理系统用到的外部接口有支付宝接口、携程(或者其他旅游网站,以携程为例)网站的接口和用户接口,下面做详述。
其中用户创建预订后在线支付需要用到支付宝接口,大多数的酒店等信息等来自于携程网站的外部接口,用户在使用该系统时提供了方便用户操作的用户接口。

\subsection{支付宝接口}
\indent    下单并支付交易接口:alipay.trade.page.pay。请求参数有商户订单号、产品码和订单总金额等参数,调用该api可以用于完成PC场景下单并支付。\newline
\indent    统一收单交易退款接口:alipay.trade.refund。当交易发生之后一段时间内,由于买家或者卖家的原因需要退款时,卖家可以通过退款接口将支付款退还给买家,支付宝将在收到退款请求并且验证成功之后,按照退款规则将支付款按原路退到买家帐号上。 请求参数有订单号、支付宝交易号等参数。
\subsection{携程接口}
\indent    酒店查询 \newline
\indent    接口地址:http://api2.juheapi.com/xiecheng/hotel/search \newline
\indent    返回格式:json \newline
\indent    请求方式:post(application/json; utf-8) \newline

\subsection{用户接口}
用户接口即为展现给用户的交互界面,用户通过web界面的提交表单或者预订选项与管理系统进行交互。

\section{内部接口}
内部模块/系统之间的交互的接口。
\subsection{酒店/火车票/航班/景点模块 与 服务器逻辑控制模块}
用户登录后可以进行酒店/火车票/航班/景点门票的查询请求,服务器端的逻辑模块提供了接收查询请求的接口,在收到提查询所有数据或者是关键词查询的请求后,服务器的逻辑控制模块会针对提交请求的类型进行相应处理,检测操作合法性以及确定相关的数据库模块,生成sql查询语句。同时,用户也可以提交预订请求,管理员登录后可以进行酒店/火车票/航班/景点/优惠券/活动推送模块的新建和更新操作,提交请求到服务器逻辑控制模块,同样的生成sql语句对数据库进行操作。

\subsection{客户服务模块 与 服务器逻辑控制模块}
用户在登录状态下可以通过网页端的客服选项发起一个对话请求,服务器逻辑控制模块为客户服务模块提供了接收用户消息请求的接口,收到用户消息后服务器可以进行调度并且将用户的消息转发给某个客服,同样的客服的消息也可以通过服务器逻辑控制模块的接口转发回建立会话的客户。
\subsection{服务器逻辑控制模块 与 数据库维护模块}
服务器在接收到客户端的请求之后,还要与数据库进行通信。服务器逻辑控制模块通过数据库维护模块提供的查询和更新接口可以高效的接收来自于服务器的相应请求,并且数据库完成操作后将数据结果返回给服务器。
\subsection{推荐系统模块 与 数据库维护模块}
由于该旅游管理系统增加了推荐系统模块,所以必须提供一个为之输入数据集的接口,而推荐系统模块和数据库维护模块之间的数据接口就可以实现这一过程,对于特定的用户,推荐系统需要获知该用户的历史行为数据,并且要结合当前数据库中可供推送的所有数据和其他用户的数据,然后推荐系统经过算法分析和学习之后将最可能符合用户喜欢的一些信息推送给用户。
