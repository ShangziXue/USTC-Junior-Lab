\chapter{数据结构设计}
\section{逻辑结构设计}
\subsection{用户管理系统数据结构设计}
讲述本系统内需要什么数据结构。这指的是程序运行过程中维护的数据结构。只是举个例子,此处应和3.3一致。

\subsection{客户端数据结构}
\indent    1.火车票模块 \\
\indent    (a)维护的数据结构:火车班次(String),座位类型(String),价格(Integer),座位数(Integer),出发时间和到达时间(DateTime),出发城市和目的城市(String),数据缓冲区,通信计时,校验串。 \\
\indent    (b)说明:该模块的请求发送和接收响应往往都是以表单的形式,根据用户实际在查看和预订火车票的需求维护以上所述的数据类型,并且为了维护数据结构,将其设为一个表单类(继承来FLaskForm)。当大量数据就进行通信传输时,需要缓冲区类型来存储图片等等数据,据结构通信计时器查看是否发生请求超时。
\\

\indent    2.航班模块 \\
\indent    (a)维护的数据结构:航班号(String),舱位(String),价格(Integer),座位数(Integer),起飞时间和到达时间(DateTime),出发城市和目的城市(String),数据缓冲区,通信计时,校验串。\\
\indent    (b)说明:该模块的请求发送和接收响应往往都是以表单的形式,根据用户实际在查看和预订航班的需求维护以上所述的数据类型,并且为了维护数据结构,将其设为一个表单类(继承来FLaskForm)。当大量数据就进行通信传输时,需要缓冲区类型来存储图片等等数据,据结构通信计时器查看是否发生请求超时。
\\ \\

\indent    3.登录注册模块 \\
\indent    (a)维护的数据结构:用户名(String),密码(String),姓名(String),年龄(Integer),性别(String),手机号(String),地址(String),通信计时,校验串。\\
\indent    (b)说明:用户在登录时需要用户名(String)和密码(Stirng),注册时需要提交的表单包含以上所有的数据类型。 
\\

\indent    4.景点模块 \\
\indent    (a)维护的数据结构:景点名(String),位置(String),景点特色(String),门票类型(String),价格(Integer),使用期限(DateTime),通信计时,校验串,数据缓冲区。\\
\indent    (b)用户在预订门票时需要浏览一个景点的详细信息,所以景点的位置和景点的特色在前端用String数据类型表示维护,具体的其他信息价格为整型、使用期间为DateTime类型。
\\

\indent    5.酒店模块 \\
\indent    (a)维护的数据结构:酒店名(String),位置(String),房间类型(String),价格(String),房间数(Integer),使用期限(DateTime),通信计时,校验串,数据缓冲区。\\
\indent    (b)将以上数据类型集合封装为一个表单类,具体如模块所述。
\\

\indent    6.优惠券模块 \\
\indent    (a)维护的数据结构:折扣商家(String),类型(String),折扣(Integer),优惠券数量(String),使用期限(DateTime),通信计时,校验串,数据缓冲区。\\
\indent    (b)将以上数据类型集合封装为一个表单类,具体如模块所述。
\\

\indent    7.活动推送模块 \\
\indent    (a)维护的数据结构:活动名(String),内容信息(String),活动价格(Integer),结束时间(DateTime),通信计时,校验串,数据缓冲区。 \\
\indent    (b)说明:活动推送内容展现在网页端的首页,在前端需要维护表述一个活动的数据类型如上。
\\ \\

\indent    8.客服模块 \\
\indent    (a)维护的数据结构:请求/接收消息(Message),数据缓冲区,通信计时,校验串 \\
\indent    (b)说明:客服模块中用户发送和接收消息的过程为一个即时通信过程,将用户发送/接收的消息封装后进行传输,并且需要动态的缓冲区存储可能到达的数据包,通信计时器和校验串为了保证通信的正常进行和数据包的完整性。
\\

\subsection{服务器端数据结构}


(1)数据库操作:
\begin{itemize}
    \item 操作表对象;
    \item 操作类型
    \item 操作值
    \item 操作语句
    \item 发生时间
    \item 是否成功
\end{itemize}


(2)表对象信息:
\begin{itemize}
    \item 表名
    \item 表记录数
    \item 表最后一次操作
    \item 表可用性
\end{itemize}


(3)服务器接收请求:
\begin{itemize}
    \item 请求来源
    \item 目标URL
    \item 请求类型
    \item 请求资源
    \item 请求源代码
    \item 发生时间
    \item 是否已回应
\end{itemize}


(4)服务器回应:
\begin{itemize}
    \item 回应目标URL
    \item 回应来源
    \item 回应信息体
    \item 回应对应请求
    \item 回应时间
    \item 是否已送达
\end{itemize}


(5)推荐模块:
\begin{itemize}
    \item 用户名
    \item 预订列表
    \item 活动列表
    \item 相似度
\end{itemize}


\section{物理结构设计}
无

% \newpage

% \section{数据结构与程序模块的关系}
% [此处指的是不同的数据结构分配到哪些模块去实现。可按不同的端拆分此表]
% \begin{table}[htbp]
% \centering
% \caption{数据结构与程序代码的关系表} \label{tab:datastructure-module}
% \begin{tabular}{|c|c|c|c|}
%     \hline
%     · & 模块1 & 模块2 & 模块3 \\
%     \hline
%     结构1 & · & Y & · \\
%     \hline
%     结构2 & · & Y & · \\
%     \hline
%     结构3 & · & Y & · \\
%     \hline
%     结构4 & Y & · & · \\
%     \hline
%     结构5 & · & · & Y \\
%     \hline
% \end{tabular}
% \note{各项数据结构的实现与各个程序模块的分配关系}
% \end{table}